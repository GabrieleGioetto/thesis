\subsection{Datasets preprocessing}
The \textit{Absenteeism at work Data Set} is the only dataset that contains data about people that is not already grouped and averaged. This means that there is a record in the dataset for each employee request, which contains the personal information of the employee, the reason of the absence and the time of absence in hours. For privacy reasons, we used a Bayesian Network. A Bayesian network is a probabilistic graphical model that measures the conditional dependence structure of a set of random variables based on the Bayes theorem. The features that we have used to build the Bayesian network are the \textit{reason for absence}, the \textit{month of absence} and the \textit{time of absence}.
\\
Using the \textit{Absenteeism at work Data Set} we learn conditional probability distributions from data, to which we add a Laplace noise for differential privacy. Then we can sample new data that follow the original distributions, but that are not equal to the original ones. The absence reasons in the dataset are given as ICD(International Classification of Diseases) codes, to make them more human readable, we picked for each ICD code the corresponding more frequent diseases.

In the \textit{National Occupational Employment and Wage Estimates United States} dataset, we sample employees based on the number of people employed in a certain field. Therefore for example since `Retail Sales Workers` consists of the 5.4\% of the total occupations, then the sampled employee will have the 5.4\% of possibility to have as occupation `Retail Sales Worker`.
The current salary of the employee is calculated adding a Gaussian noise to the average salary of the employee's occupation, and then the salary raise requested is picked randomly between 5\%-10\%.

The ranges of wage gap, used in the tickets regarding explanation for the gender wage gap in the company, are sampled randomly from the dataset \textit{Gender pay gap in the UK}, adding a Gaussian noise for privacy reasons.

To sample the cities for the requests of accomodation, we randomly sample from all the cities with more than 100,000 inhabitants from the country of residence of the synthetic employee. To calculate the duration of the accommodation we pick a random number of months between 1 and 12.

To get data for the category type `refund travel`, we sample randomly one flight from all the flights leaving from the country of the synthetic employee. The data are taken from the \textit{OpenFlights database}.

The complaints about coworkers and superiors and the life events that can affect the work life of a person were handcrafted.

\subsection{Bayesian Network}
A Bayesian Model is a machine learning algorithm that is built with Bayes Theorem in mind.  It is based on the probability theory and it enables a machine to learn from the data.
\begin{equation}
    p(A|B) = \frac{p(B|A)p(A)}{p(B)}
\end{equation}
Formally, the Bayesian network is a directed graph G = (V,E) with
\begin{itemize}
    \item A feature for each node i appartenente a V
    \item A conditional probability distribution for each edge, so the edge from feature $i$ to $j$ represents $p(x_j| x_i)$
  \end{itemize}
The base version of a Bayesian network works with discrete variables, however there are also implementations that consider also continuous variables [https://www.jair.org/index.php/jair/article/download/11063/26242/] \\
Building a Bayesian network starting from the \textit{Absenteeism at work Data Set} is relatively easy, we calculate the likelihood distribution $p(x\_i|x\_j) \forall x_i, x_j \in D$ for each other feature $x_j$. As a prior we used a dirichlet distribution. 
[https://mbernste.github.io/files/notes/Psuedocounts.pdf]
https://cs.nyu.edu/~roweis/csc412-2004/notes/lec13x.pdf
We then added pseudocounts to the observed counts in the data used to calculate $p(x_j| x_i)$. This technique is used to diminish the overfitting of data. The values we used for pseudocount is $\gamma=1$. \\
Since we learn the conditional probability distribution from our data, the structure of the network or the conditional probabilities may therefore leak some information on an individual in the dataset. In order to provide strong privacy guarantees and minimize
the re-identification risk, we leverage the notion of differential-privacy: we perturb the data adding a noise sampled from a Laplace distribution 
\begin{equation}
    z \sim Laplace \left(0, \frac{2 \cdot n_{features}}{\gamma \cdot \epsilon} \right)
\end{equation}
where $\epsilon$ is the privacy budget for differential privacy, which controls the anonymization level.

\begin{equation}
    Pr[M(X) \in Z] \leq e^{\epsilon} \cdot Pr[M(X') \in Z]
\end{equation}
Once the private Bayesian network is built, we can sample new values for all the nodes in the graph. These generated values will have the same distribution and preserve the consistency and statistical properties of the original dataset up to the noise addition which acts as a de-identification barrier. 

