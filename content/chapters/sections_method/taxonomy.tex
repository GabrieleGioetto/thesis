\section{Taxonomy}
\label{sec:taxonomy}
\todo{I suggest to explain the overall process in this chapter, for what purpose are we defining a taxonomy? what is the overall idea? on a stylistic basis, I would start with text and have pictures after their description.} 
When building a dataset is important to decide the features that will be used. Features should be chosen carefully to ensure that the dataset is relevant and useful. In most HR ticketing systems, the tickets belong to a category, which helps the HR department to classify the tickets and send them to the right person to answer them. \\
Usually, HR tickets can belong to various categories, which can range from a request for a shift change to a complaint about a colleague. \\
We built a taxonomy of tickets, which is structured into categories and subcategories. The categories and subcategories let us define a finite set of possible topics of the tickets and give a precise structure to the dataset. \\
Each subcategory has its own variables that are used as inputs for the tickets' generation. The variables are sampled from real-world datasets. \\
For example, to create a request for sick leave, we pass as inputs to our model the reason and the number of days of sick leave requested, which will be acquired from an external dataset. \\
We have a selection of templates and prompts to kick-start the generation process. Every category has its own distinct templates and prompts. The taxonomy ( Shown in \autoref{fig:taxonomy} ) has been inspired by industrial approaches on the categorization of most common HR requests\todo{I would not say this, it could be good to find references in HR books. If not possible, I would say, ``inspired by industrial approaches on the categorization of most common HR requests.''}, however, the final taxonomy ( Shown in \autoref{fig:final_taxonomy} ) presented here is a subset of the original one due to the unavailability of public data on certain topics ( Ex. \textit{Work benefits}) \\ The final complete taxonomy and the complete list of all variables used for each category/sub-category can be seen in the \autoref{table:categoriesTable}

\begin{figure}[h] 
    \includegraphics[width=\textwidth]{images/Taxonomy_Tickets.drawio.png}
    \caption{Initial Taxonomy}
    \label{fig:taxonomy}
\end{figure}    

\begin{figure}[h] 
    \includegraphics[width=\textwidth]{images/Taxonomy_Tickets_implemented.drawio.png}
    \caption{Final taxonomy, reduced due to unavailability of data}
    \label{fig:final_taxonomy}
\end{figure}    


\begin{table}[h]
    \resizebox{\textwidth}{!}{
    \begin{tabular}{|l|l|l|}
    \hline
        Category            & Sub-category   & variables                                   \\ \hline
        Ask Information     & Accommodation   & location, duration                          \\
        Complaint           & About coworker & complaint, reason                           \\
                            & About superior & complaint, reason                           \\
        Timetable change    & Shift change   & reason\_of\_change, old\_date, new\_date    \\
        Salary              & Salary raise   & old\_salary, new\_salary, increase,work\_title \\
                            & Gender pay gap & wage\_gap                                   \\
        Life Event          & Health issues  & disease, number\_of\_days\_of\_sick\_leave  \\
                            & Personal issues  & issue, number\_of\_days                   \\
        Refund              & Travel         & from, to, date\_travel                     \\
    \hline
    \end{tabular}
    }
\caption{Table of all defined categories and sub-categories with their respective variables}\label{table:categoriesTable}
\end{table}