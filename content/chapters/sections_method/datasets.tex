\section{Datasets}
In order to generate tickets, we decided to use real data as a starting point to make them as much realistic as possible. Another reason to use real data is that it makes the dataset useful for use cases such as anonymization. \\
The datasets that we used are all public and available online. In some cases where no datasets were available, we created them manually from personal experience. \\
The datasets are:
\begin{itemize}
    \item \textit{Absenteeism at work Data Set}: contains records of work absences, with the reason for the absence (almost always a disease) and the number of hours of absence. It is used to create requests for days off for health reasons
    \item \textit{National Occupational Employment and Wage Estimates United States}: estimates of wages in the US calculated with data collected from employers in all industry sectors in metropolitan and nonmetropolitan areas in every state and the District of Columbia. It is used to create requests for salary raises.
    \item \textit{List of events of life}: list of major events in life. It is used to create requests for time off due to personal reasons. 
    \item \textit{Gender pay gap in the UK}: dataset of employers with 250 or more employees, comparing men's and women’s average pay across the organizations. It is used to create the requests for explanations for the wage gap between genders.
    \item \textit{OpenFlights database}: datasets of airports and flights all over the world. It is used for requests for refunds of travel.
    \item \textit{Geonames all cities with a population over 1000}: datasets of all cities of the world with a population over 1000 people. It is used for requests for information about accommodation.
\end{itemize}