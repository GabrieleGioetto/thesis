\chapter*{Abstract}

An HR (Human Resource) department in a large organization receives inquiries\slash requests from employees on multiple topics, quite different from one another. As an example, an employee can send requests dealing with health conditions, compensation/taxation, events of life (marriage, death of a relative\dots). \\
These data can be used for many different queries that can be useful for analysis purposes (Example: `How many people have had COVID during 2021`). However, HR tickets typically contain personal data, that cannot be processed without the consent of the data subject according to the European privacy regulation (GDPR). \\
To be able to process documents with personal data, we can identify the pieces of information that qualify as personal data in a communication and subsequently anonymize such information using the appropriate techniques.
A significant part of this problem is represented by the complex nature of personal data according to GDPR:\@ personal data are defined as `\textit{any piece of information that can be connected to an identified or identifiable natural person}` It comprises obvious identifiers like social security numbers, email addresses but also, elements like `the Italian intern working for SAP in South of France`.
To the best of our knowledge, it does not exist a public dataset of HR tickets that can be used to train machine learning models, the main reason being the difficult nature of these types of data. Synthetic data, which are artificial data that are generated from original data and a model that is trained to reproduce the characteristics and structure of the original data, follow a data protection by design approach.
To address the need for a large dataset of HR tickets, we created a taxonomy of tickets, we found real data that can be used as support to create synthetic tickets and developed Ticket Generator: an application that can produce as many tickets as needed belonging to different categories, we released a dataset of previously created tickets and we showcase some possible use cases of the dataset.
