\chapter{Related Works}
\label{sec:RelatedWorks}

As far as we know, it has never been published a dataset of HR tickets written by company employees, due to the sensitive nature of the data. We believe this is the case not only because of the GDPR laws, but also because these data can be damaging both for the company and the employees in some cases ( For example an employee criticizing the working environment or leaking some personal information in a ticket ).  \\
However, some datasets can resemble what we are trying to build, with similar language style and intents:
\begin{itemize}
    \item \textit{Consumer Complaint}: Consumer Financial Protection Bureau's online database of customer complaints about different financial products. The overall dataset contains over 600,000 complaints and each record has the complaint's text description, the product that is the cause of the complaint, the company which issues the product, and the category of the issue. In the dataset the personal information are masked.
    \item \textit{Enron Mail}: The \textit{Enron Mail} dataset contains about 0.5M emails of 150 senior management from the Enron corporation. This data was originally made public, and posted to the web, by the Federal Energy Regulatory Commission during its investigation for fraud. The corpus is one of the only few publicly available mass collections of real emails easily available for study.
    \item \textit{TAB}: the Text Anonymization Benchmark corpus\cite{pilan2022text} is a privacy-oriented annotated text resource. The corpus comprises 1,268 English-language court cases from the European Court of Human Rights (ECHR) enriched with comprehensive annotations about the personal information appearing in each document, including their semantic category, identifier type, confidential
    attributes, and co-reference relations.
\end{itemize}
