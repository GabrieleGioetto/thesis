\chapter*{Abstract French}

Le service RH (Ressources Humaines) d'une grande entreprise reçoit des demandes de renseignements de la part des employés sur de multiples sujets, très différents les uns des autres. Par exemple, un employé peut envoyer des demandes concernant l'état de santé, la rémunération/imposition, les événements de la vie (mariage, décès d'un proche\dots). \\
Ces données peuvent être utilisées pour de nombreuses requêtes différentes qui peuvent être utiles à des fins d'analyse (Exemple : `Combien de personnes ont eu un COVID au cours de l'année 2021`). Cependant, les tickets RH contiennent généralement des données personnelles, qui ne peuvent être traitées sans le consentement de la personne concernée, conformément au Règlement Général sur la Protection des Données (RGPD). \\
Pour pouvoir traiter les documents contenant des données à caractère personnel, nous pouvons identifier les éléments d'information qui sont qualifiés de données à caractère personnel dans une communication et, par la suite, rendre ces informations anonymes en utilisant les techniques appropriées.
Une partie importante de ce problème est représentée par la nature complexe des données personnelles selon le RGPD:\@ les données personnelles sont définies comme `\textit{  toute information se rapportant à une personne physique identifiée ou identifiable}`. Elles comprennent des identifiants évidents comme les numéros de sécurité sociale, les adresses e-mail, mais aussi des éléments comme ``le stagiaire italien travaillant pour SAP dans le sud de la France``.
À notre connaissance, il n'existe pas de base de données publique des tickets RH qui peuvent être utilisée pour entraîner des modèles d'apprentissage automatique, principalement en raison de la nature difficile de ces types des données. Les données synthétiques, qui sont des données artificielles générées à partir de données originales et d'un modèle entraîné à reproduire les caractéristiques et la structure des données originales, suivent une approche de protection des données par conception.
Pour répondre au besoin d'un grand ensemble de données de tickets RH, nous avons créé une taxonomie de tickets, nous avons trouvé des données réelles qui peuvent être utilisées comme support pour créer des tickets synthétiques et nous avons développé \textit{Ticket Generator}: une application qui peut produire autant de tickets que nécessaire appartenant à différentes catégories, nous avons publié un ensemble de tickets précédemment créés et nous présentons quelques cas d'utilisation possibles de l'ensemble de données.